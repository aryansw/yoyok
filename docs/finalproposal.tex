\documentclass[review,twocolumn,preprint]{sigplanconf}

\bibliographystyle{abbrvnat}

\begin{document}

\title{Formal Semantics of Linear Types for Stackful Coroutines: Final Proposal}
\authorinfo{Aryan Wadhwani}{Purdue University}{wadhwani@purdue.edu}
\date{}

\maketitle


\section{Introduction}

\emph{Four to five \citep{krust2018} (one column) introducing the problem you
  plan to investigate}


\section{Background}

\emph{One page (two colunns) of background/related work.  This should
  include descriptions of anywhere from 5 to 10 papers or tools that motivate
  the  problem presented in the introduction.}


\section{Motivating Example}

\emph{Four to five paragraphs (one column) Present a detailed example
  illustrating the problem, a provide an informal overview of the
  solution.}

\section{Approach}

\emph{Two pages (4 columns) detailing your approach.  This should
  include a detailed description of (a) methodology, (b) pseudo-code
  describing your algorithm or technique, and (c) how the approach
  addresses the motivating example}

\emph{Include a figure that illustrates your idea.  This could be an
  architecture or workflow diagram, expected performance curve; the
  idea of the figure is to illustrate the main elements of the project.}

\section{Milestones}

\emph{One page (2 columns) with details on your current implementation
  and status, open issues, potential risks, etc.}

\bibliography{../lastproposal}

\end{document}